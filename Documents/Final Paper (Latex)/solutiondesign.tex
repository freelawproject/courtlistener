\label{solutiondesign}
\section{Designing a solution}
When initially creating the design of this system, I began by seeking out as many legal technologists, lawyers, academics, and librarians as possible. In addition to numerous informal conversations, I, with the assistance of others, completed a total of seven formal 1-on-1 interviews and two group interview sessions. The seven 1-on-1 sessions were approximately 45 minutes in length, and interviewed people of the following professions:
\begin{itemize}
\item Law professor: 1 
\item Practicing lawyer: 1
\item Law technologist: 1
\item Law librarians: 3
\item Private legal researcher: 1
\end{itemize}
In addition, group interviews were completed with the legal team at the Electronic Frontier Foundation (EFF) and with the University of California, Berkeley, School of Law legal librarians group. These interviews proved invaluable to my understanding of the problem space, and legal research on the whole. In addition to helping me shape the scope of the project, these interviews allowed me to bounce ideas off of the people that would likely use or recommend the product, and who were experts in the legal research field. The questions I asked during these interviews attempted to teach me about their day to day work and the motivations for any processes they have or tools they use. 

At the time of the interviews, user participation was considered as a method of accurately and efficiently categorizing and creating content, however a significant finding from the interviews was that nearly all of the people we interviewed felt that their time for researching was severely constrained, and that there was little that a website could do that would motivate them to contribute. To a prompt regarding whether people would contribute to a system if it meant creating a public good, one public interest attorney expressed that, ``For people to contribute, it would have to benefit them.'' By this comment, he expressed his opinion that contributing to a public good would not be sufficient motivation for busy lawyers, and that any contribution would have to directly benefit the person making it. This is a common sentiment among users, and creating systems in which users feel that their work towards the public good is also for their own good is indeed challenging. As a result of this finding however, ideas for user-contributed content were set aside, and the site was designed to be exclusively unidirectional with regards to content production and curation.

A notable person that I interviewed was a private legal researcher. As a part of her job, each morning she spends an hour or two researching new cases. Generally, she looks for two different items while researching. First, she attempts to identify any new opinions from California courts that could be relevant to the firm where she works, and second, she looks for any ongoing cases that are in her area of the law. I discussed with her the tools she uses, and discovered that for the most part, she relies on curated electronic email lists, as discussed in section 1, and on browsing court websites manually, traipsing for hints of relevant cases. She was very excited to hear about the planned platform, but disappointed that it would initially only contain records for federal cases since her area of research was state law.

During the other interviews, I attempted to learn more specific details about the kinds of expectations users will have when approaching a new research tool. Some questions that aimed to answer were whether users would be comfortable with Boolean searching, what kinds of Boolean connectors they might find valuable, whether they use RSS feeds, and the kinds of document categorization they might expect. The result of these inquiries indicated that the primary users of this tool are highly sophisticated users. Most of the people interviewed knew about or used RSS feeds, and all of them were familiar with Boolean connectors. When speaking to the EFF legal team, we were able to determine which connectors people valued.\footnote{Specifically, they mentioned: Number of word occurrences, sentence and paragraph containment, and quorum identification (e.g. find two of the eight following words).} Most of their requests are now possible on CourtListener.com.

As for the kinds of categorization users wanted, the interviews revealed that users felt that more categorization was always better. At the time of the interviews, consideration was given to creating a system that semantically analyzed, and automatically extracted and categorized a court opinion along a range of categories such as the judges, legal domain, precedential nature, plaintiffs, defendants, case name and case number. Ultimately, most of these categories were not implemented, however the case name, number, date and precedential status are all obtained and placed in the database.

Other design considerations that were made early on were that a clean and simple interface was a must, and that the site itself must have minimal visual clutter, with as much standards-compliance and accessibility as possible. These decisions were made in an effort to make the site as useful as possible to as many people as possible, and to minimize visual distractions, making users more efficient.

From these findings, and the above design considerations, the design of the site proceeded along two paths. First, a so-called ``MoSCoW'' document was drawn up that contained lists of the things the site Must, Should, Could, and Wouldn't do.\cite{clegg_case_1994} This document served the purpose of listing and prioritizing all the ideas that were on the table for the project. The second path that was followed was translating the emerging MoSCoW analysis into a database model, URL design, and interface sketches.\footnote{See appendix I for details.} Once these plans were created, designing and building the site was largely a matter of choosing and implementing appropriate technology solutions.
